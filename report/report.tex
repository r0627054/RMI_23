\documentclass{ds-report}

\assignment{JAVA RMI} % Set to `Java RMI`, `Java EE` or `Google App Engine`.
\authorOne{Dries Janse} % Name of first team partner.
\studentnumberOne{r06} % Student number of first team partner.
\authorTwo{Steven Ghekiere} % Name of second team partner.
\studentnumberTwo{r0626062}  % Student number of second team partner.

\begin{document}
	\maketitle

	\paragraph{1. How would a client complete one full cycle of the booking process, for both a successful and
failed case? Base yourself on the example scenarios in Figure 1. Create sequence drawings to
illustrate this.} 
	Answer 1
	
	\paragraph{2. When do classes need to be serializable? You may illustrate this with an example class.} 
	Answer 2

	\paragraph{3. When do classes need to be remotely accessible (Remote)? You may illustrate this with an example
class.} 
	Answer 2

	\paragraph{4. What data has to be transmitted between client and server and back when requesting the number
of reservations of a specific renter?} 
	Answer 2

	\paragraph{5. What is the reasoning behind your distribution of remote objects over hosts? Show which
hosts execute which classes, if run in a real distributed deployment (not a lab deployment where
everything runs on the same machine). Create a component/deployment diagram to illustrate this:
highlight where the client and server are.} 
	Answer 2

	\paragraph{6. How have you implemented the naming service, and what role does the built-in RMI registry play?
Why did you take this approach?} 
	Answer 2

	\paragraph{7. Which approach did you take to achieve life cycle management of sessions? Indicate why you
picked this approach, in particular where you store the sessions.} 
	Answer 2

	\paragraph{8. Why is a Java RMI application not thread-safe by default? How does your application of synchronization achieve thread-safety?} 
	Answer 2

	\paragraph{9. How does your solution to concurrency control aspect the scalability of your design? Could
synchronization become a bottleneck?} 
	Answer 2

	
	\clearpage
	
	% You can include diagrams here.
	
\end{document}